%%%%%%%%%%%%%%%%%%%%%%%%%%%%%%%%%%%%%%%%%%%%%%%%%%%%%%%%%%%%%%%%%%%%%%
%     File: ExtendedAbstract_imple.tex                               %
%     Tex Master: ExtendedAbstract.tex                               %
%                                                                    %
%     Author: Andre Calado Marta                                     %
%     Last modified : 27 Dez 2011                                    %
%%%%%%%%%%%%%%%%%%%%%%%%%%%%%%%%%%%%%%%%%%%%%%%%%%%%%%%%%%%%%%%%%%%%%%
% A Calculation section represents a practical development
% from a theoretical basis.
%%%%%%%%%%%%%%%%%%%%%%%%%%%%%%%%%%%%%%%%%%%%%%%%%%%%%%%%%%%%%%%%%%%%%%

\section{Collider experiments}
\label{sec:det}

\subsection{The LHC and the ATLAS detector}
 
The Large Hadron Collider (LHC) is housed by the European Organization for Nuclear Reasearch (CERN) and located beneath the Franco-Swiss boarder in the Geneva area. It consists of a $27$ km ring dedicated (most of the time) to delivering proton-proton collisions at a center of mass (CM) energy of $\sqrt{s}=13$ TeV. The two general purpose experiments, ATLAS and CMS, have a broad experimental physics program that includes searches for new physics. The LHCb experiment is dedicated to the study of beauty particles and the ALICE experiment is optimized to study heavy ion collisions. 

The ATLAS detector is a multipurpose particle physics apparatus with forward-backward symmetric  cylindrical geometry. A combination of cartesian and cylindrical coordinates is used to describe it. The origin is defined to coincide with the interaction point. The Cartesian system is right-handed and the z axis is defined to be the direction of the beam. The x-axis points from the interaction point to the center of the LHC ring and the y-axis points upwards. The azimuthal angle, $\phi$, is measured around the beam axis and the polar angle, $\theta$, from the beam line. The pseudorapidity is defined as $\eta = − \ln \tan(\theta/2)$. The inner tracking detector (ID) consists of  a  silicon  pixel detector,  a  silicon  microstrip detector, and a straw-tube transition radiation tracker. It is contained in a superconducting solenoid magnet that provides a $2$ T magnetic field and surrounded by a high-granularity liquid-argon sampling electromagnetic calorimeter (ECAL). The ECAL covers the pseudo-rapidity range $|\eta|<3.2$. The hadronic calorimetry in the pseudorapidity range $|\eta| < 1.7$ is provided by a scintillator-tile calorimeter (TileCal). For $|\eta|>1.5$ liquid-argon calorimeters extend the pseudorapidity range to $|\eta|=4.9$. The LAr calorimeter is divided in end-cap and forward. These cover the pseudorapidity ranges $1.5 < |\eta| < 3.2$ and $3.2 < |\eta| <
4.9$. In the end-cap the segmentation is $\Delta\eta\times\Delta\phi = 0.1 \times 0.1$ for $1.5 < |\eta| < 2.5$ and $0.2 \times 0.2$ for $2.5 < |\eta| < 3.2$. In the forward region the segmentation is $\Delta\eta\times\Delta\phi = 0.2 \times 0.2$. The muon spectrometer (MS) surrounds the calorimeters and it is the outermost layer of the detector. It is composed of Monitored Drift Tubes and Cathode Strip Chambers.

\subsection{The FCC-hh}

The FCC-hh baseline design consist of a of a proton-proton circular collider with a maximum CM energy of $\sqrt{s}=100$ TeV housed by a $100$ km tunnel in the area of Geneva. It will deliver a peak luminosity of $\mathcal{L}=30\times 10^{34}~\text{cm}^{-2}\text{s}^{-1}$ in its ultimate phase which will result in a $O(30)~\text{ab}^{-1}$ per experiment. This machine will extend the research program of the LHC (and of the HL-LHC) after these have reached their full discovery potential, by around 2040.

The design of the FCC-hh baseline detector has been greatly based on that of the ATLAS and CMS experiments, in particular the central barrel. The layers and sub detectors are arranged in the same order and make use of very similar technologies. The ID detector covers the pseudorapidity range $|\eta|<6$ and it will be instrumented with pixel and strip detectors. The ECAL covers the pseudorapidity range $|\eta|<6$. The proposed layout is a LAr sampling configuration with lead, glue and steal plates as absorbers. The granularity is expected to be two to four times better than for the ATLAS ECAL. The hadronic calorimeter covers the pseudorapidity range $|\eta| < 6$. It is divided in barrel, end-cap and forward that cover the pseudorapidity ranges $|\eta| < 1.3$, $1.0 < |\eta| < 1.8$ and $2.3 < |\eta| < 6.0$, respectively. For the barrel and end-cap calorimeters, the expected segmentation $\Delta\eta\times\Delta\phi = 0.025 \times 0.025$ while for the forward calorimeter it is $\Delta\eta\times\Delta\phi = 0.05 \times 0.05$. Overall, this corresponds to approximately four times the ATLAS HCAL granularity. The MS cover the pseudorapidity range $|\eta|<6$ and it consists of a layered structure of gas chambers.