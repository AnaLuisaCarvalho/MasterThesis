%%%%%%%%%%%%%%%%%%%%%%%%%%%%%%%%%%%%%%%%%%%%%%%%%%%%%%%%%%%%%%%%%%%%%%
%     File: ExtendedAbstract_imple.tex                               %
%     Tex Master: ExtendedAbstract.tex                               %
%                                                                    %
%     Author: Andre Calado Marta                                     %
%     Last modified : 27 Dez 2011                                    %
%%%%%%%%%%%%%%%%%%%%%%%%%%%%%%%%%%%%%%%%%%%%%%%%%%%%%%%%%%%%%%%%%%%%%%
% A Calculation section represents a practical development
% from a theoretical basis.
%%%%%%%%%%%%%%%%%%%%%%%%%%%%%%%%%%%%%%%%%%%%%%%%%%%%%%%%%%%%%%%%%%%%%%

\section{The LHC and the ATLAS detector}
\label{sec:det}
 
The Large Hadron Collider (LHC) is housed by the European Organization for Nuclear Reasearch (CERN) and located beneath the Franco-Swiss boarder in the Geneva area. It consists of a $27$ km ring dedicated (most of the time) to delivering proton-proton collisions at a center of mass (CM) energy of $\sqrt{s}=13$ TeV. The two main purpose experiments, ATLAS and CMS, are dedicated to the search for any hints of new physics. The LHCb experiment is dedicated to the study of beauty particles and the ALICE experiment is optimized to study heavy ion collisions. 

The ATLAS detector is a multipurpose particle physics apparatus with forward-backward symmetric  cylindrical geometry. The inner tracking detector (ID) consists of  a  silicon  pixel detector,  a  silicon  microstrip detector, and a straw-tube transition radiation tracker. It is contained in a superconducting solenoid magnet that provides a $2$ T magnetic field and surrounded by a high-granularity liquid-argon sampling electromagnetic calorimeter (ECAL). The ECAL covers the pseudo-rapidity range $|\eta|<3.2$. The hadronic calorimetry in the pseudorapidity range $|\eta| < 1.7$ is provided by a scintillator-tile calorimeter (TileCal). For $|\eta|>1.5$ liquid-argon calorimeters extend the pseudorapidity range to $|\eta|=4.9$. The LAr calorimeter is divided in end-cap and forward. These cover the pseudorapidity ranges $1.5 < |\eta| < 3.2$ and $3.2 < |\eta| <
4.9$. In the end-cap the segmentation is $\Delta\eta\times\Delta\phi = 0.1 \times 0.1$ for $1.5 < |\eta| < 2.5$ and $0.2 \times 0.2$ for $2.5 < |\eta| < 3.2$. In the forward region the segmentation is $\Delta\eta\times\Delta\phi = 0.2 \times 0.2$. The muon spectrometer (MS) surrounds the calorimeters and it is the outermost layer of the detector. It is composed of Monitored Drift Tubes and Cathode Strip Chambers.