%%%%%%%%%%%%%%%%%%%%%%%%%%%%%%%%%%%%%%%%%%%%%%%%%%%%%%%%%%%%%%%%%%%%%%
%     File: ExtendedAbstract_concl.tex                               %
%     Tex Master: ExtendedAbstract.tex                               %
%                                                                    %
%     Author: Andre Calado Marta                                     %
%     Last modified : 27 Dez 2011                                    %
%%%%%%%%%%%%%%%%%%%%%%%%%%%%%%%%%%%%%%%%%%%%%%%%%%%%%%%%%%%%%%%%%%%%%%
% The main conclusions of the study presented in short form.
%%%%%%%%%%%%%%%%%%%%%%%%%%%%%%%%%%%%%%%%%%%%%%%%%%%%%%%%%%%%%%%%%%%%%%

\section{Conclusions}
\label{sec:concl}

This article presents a feasibility study targeting the search for Higgs pair production in the $pp\rightarrow hh\rightarrow b\overline{b}b\overline{b}$ channel at a center of mass energy of $\sqrt{s}=100$ TeV. The analysis is based on rectangular cuts on kinematic and substructure variables, and targets a boosted event topology. The main backgrounds are QCD multijet and $t\overline{t}$ production. The impact of the granularity of the hadronic calorimeter on the achieved significance was studied. 

Three signal models are investigated: SM, $1$ TeV DM spin-$0$ mediator and CP-conserving type II 2HDM with $m_H=900$ GeV. Due to the small cross section, the DM mediator model originates a significance of the order of $2$, making it a very challenging and probably unaccessible background. 

For the FCC-hh, the significances achieved for the SM and 2HDM signal modes were $(S/\sqrt{B})_{SM}=8.8\pm 1.6~\text{(stat.)}~^{+4.4}_{-3.4}~\text{(sys.)}$ and $(S/\sqrt{B})_{2HDM}=16.9\pm 3.0~\text{(stat.)}~^{+8.5}_{-6.6}~\text{(sys.)}$, respectively, for an integrated luminosity of $\mathcal{L}=30~\text{ab}^{-1}$. These values are above the $5\sigma$ threshold which indicates that the total dataset that is expected to be collected at the FCC-hh should be enough to observe (or exclude) the production of Higgs pairs in these models.

For particle-flow jets, the efficiency of the signal increases as the granularity of the HCAL increases. It varies by approximately $30\%$ between the ATLAS detector configuration and the detector configuration with twice the granularity of the FCC baseline detector in $\eta$ and $\phi$. It is harder to discern such a clear tendency for the significance because the statistical fluctuations are large, which is due to the large statistical weight of the multijet background. Nonetheless, for particle flow (calorimeter) jets, the significance varies by approximately $30\%~(70\%)$, for the SM signal model. 

As future work, performing similar studies using different benchmark processes and full detector simulation seems to be of the utmost importance in order to optimize the design of future detectors.

