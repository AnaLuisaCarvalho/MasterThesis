%%%%%%%%%%%%%%%%%%%%%%%%%%%%%%%%%%%%%%%%%%%%%%%%%%%%%%%%%%%%%%%%%%%%%%
%     File: ExtendedAbstract_intro.tex                               %
%     Tex Master: ExtendedAbstract.tex                               %
%                                                                    %
%     Author: Andre Calado Marta                                     %
%     Last modified : 27 Dez 2011                                    %
%%%%%%%%%%%%%%%%%%%%%%%%%%%%%%%%%%%%%%%%%%%%%%%%%%%%%%%%%%%%%%%%%%%%%%
% State the objectives of the work and provide an adequate background,
% avoiding a detailed literature survey or a summary of the results.
%%%%%%%%%%%%%%%%%%%%%%%%%%%%%%%%%%%%%%%%%%%%%%%%%%%%%%%%%%%%%%%%%%%%%%

\section{Introduction}
\label{sec:intro}

- Higgs pair production as a benchmark physics process for HL and future colliders (discovery inaccessible at the LHC)\\
- Boosted analysis shows better sensitivity because QCD background can be kept under control \\
- When using boosted jets we need to resolve their substructure (jet substructure observables)\\
- The granularity of the (hadronic) calorimeter is a key parameter that determines how well we can resolve the substructure \\
- The final state with four b quarks benefits from the largest BR and is a source of fully hadronic boosted Higgs jets \\
- Goals of the work: estimate analysis sensitivity to Higgs trilinear coupling and study the influence of the HCAL granularity in the analysis
