%%%%%%%%%%%%%%%%%%%%%%%%%%%%%%%%%%%%%%%%%%%%%%%%%%%%%%%%%%%%%%%%%%%%%%
%     File: ExtendedAbstract_imple.tex                               %
%     Tex Master: ExtendedAbstract.tex                               %
%                                                                    %
%     Author: Andre Calado Marta                                     %
%     Last modified : 27 Dez 2011                                    %
%%%%%%%%%%%%%%%%%%%%%%%%%%%%%%%%%%%%%%%%%%%%%%%%%%%%%%%%%%%%%%%%%%%%%%
% A Calculation section represents a practical development
% from a theoretical basis.
%%%%%%%%%%%%%%%%%%%%%%%%%%%%%%%%%%%%%%%%%%%%%%%%%%%%%%%%%%%%%%%%%%%%%%

\section{Simulation setup}
\label{sec:sim}

The main backgrounds for di-Higgs searches in the $b\overline{b}b\overline{b}$ are multijet and $t\overline{t}$ production. All other sources of background, including processes involving Higgs bosons, are found to be negligible [REF]. In addition to these, we also consider the irreducible background as a separate sample.

We simulate the signal and background Monte Carlo samples using a fast simulation workflow. MadGraph5 aMC@NLO is used to compute the matrix elements of a given process. Showering and hadronization of colored particles are handled by Pythia8 and the detector response is parameterized using Delphes3. 

The irreducible background is generated with an extra jet with $p_T>200$ GeV at generator level. This guarantees that the pairs of b quarks are boosted enough to increase the probability of being reconstructed as a single jet. The multijet background is simulated as $jj+0/1/2~j$ where $j$ stands for a light or b jet. To make the simulation more efficient, this background is generated in several $H_T$ regions where $H_T$ is the scalar sum of the $p_T$ of all the partons at generator level. The $t\overline{t}$ background is simulated as $t\overline{t}+0/1/2~j$. 