%%%%%%%%%%%%%%%%%%%%%%%%%%%%%%%%%%%%%%%%%%%%%%%%%%%%%%%%%%%%%%%%%%%%%%
%     File: ExtendedAbstract_backg.tex                               %
%     Tex Master: ExtendedAbstract.tex                               %
%                                                                    %
%     Author: Andre Calado Marta                                     %
%     Last modified : 27 Dez 2011                                    %
%%%%%%%%%%%%%%%%%%%%%%%%%%%%%%%%%%%%%%%%%%%%%%%%%%%%%%%%%%%%%%%%%%%%%%
% A Theory section should extend, not repeat, the background to the
% article already dealt with in the Introduction and lay the
% foundation for further work.
%%%%%%%%%%%%%%%%%%%%%%%%%%%%%%%%%%%%%%%%%%%%%%%%%%%%%%%%%%%%%%%%%%%%%%

\section{State of the art}
\label{sec:backg}

%- HL upgrades for the LHC \\
%- FCC-hh, 100 TeV hadronic collider at the stage of CDR \\
%- Feasibility studies/previous work on this channel (slightly longer literature review)\\

The searches performed so far for di-Higgs production covered different decay channels and targeted not only the SM production but also some BSM scenarios where this process is enhanced. Neither could achieve enough statistical significance to declare the observation of this process nor found any deviation from the SM predictions. 

The most stringent limit comes from a combination of searches using up to $36.1~\text{fb}^{-1}$ of proton-proton collision data at a center of mass energy $\sqrt{s}=13$ TeV recorded with the ATLAS detector. The combination is performed using the analysis searching for $hh\rightarrow b\overline{b}b\overline{b}$, $hh\rightarrow b\overline{b}\tau^+\tau^.$ and $hh\rightarrow b\overline{b}\gamma\gamma$. The combined observed (expected) limit on the non-resonant Higgs boson pair cross-section is $0.22$ pb ($0.35$ pb) at $95\%$ confidence level, which corresponds to $6.7 (10.4)$ times the predicted SM cross-section. The ratio of the Higgs boson self-coupling to its SM expectation ($k_{\lambda}=\lambda_{hhh}/\lambda_{hhh}^{SM}$) is observed (expected) to be contrained at $95\%$ CL to $-5.0<k_{\lambda}<12.1 (-5.8<k_{\lambda}<12.0)$.

Monte Carlo studies assessing the feasibility of searches for di-Higgs production at the High Luminosity LHC (HL-LHC) and at the FCC-hh have been performed. For the HL-LHC, a study including the $pp\rightarrow b\overline{b}b\overline{b}$, $pp\rightarrow b\overline{b}jj$, $pp\rightarrow jjjj$ and $pp\rightarrow t\overline{t}jjjj$ reports a significance of $S/\sqrt{B}=3.1~(1.0)$ for an integrated luminosity of $3000~(300)~\text{fb}^{-1}$, considering a mean pileup of $80$. The analysis is performed in three orthogonal signal categories (resolved, intermediate and boosted) and the reported significance is obtained from the combination of the three regions. This study makes use of an artificial neural network to further increase the signal-background separation. Similar studies performed by ATLAS and CMS find the $hh\rightarrow b\overline{b}\gamma\gamma$ channel to be the most sensitive to the Higgs trilinear coupling. ATLAS reports a significance of $1.06$ for an integrated luminosity of $3000~\text{fb}^{-1}$, which translates to a $95\%$ CL limit on the ratio of the Higgs boson self-coupling to its SM expectation of $-0.8<k_{\lambda}<7.7$. This analysis is purely cut based and a mean pileup of $200$ is considered. 

For the FCC-hh, a recent study simulates the signal with an extra jet at generator level: $pp\rightarrow b\overline{b}b\overline{b}j$. The extra jet boosts the Higgs pair favoring a highly boosted virtual Higgs decaying to a pair of Higgs bosons which enhances the sensitivity to the Higgs trilinear coupling. Only the irreducible background, $pp\rightarrow b\overline{b}b\overline{b}j$, is considered. A significance of $6.61$ is reported for an integrated luminosity of $30~\text{ab}^{-1}$. No MVA techniques are applied nor pileup contribution considered.

Studies on the impact of the granularity of the calorimeters in the spatial resolving power of hadronic showers and on the resolution of jet mass and substructure variables greatly influenced the baseline design of the FCC-hh. For two Kaons with an energy of 100 GeV each and with a truth level separation equal to $0.035$ it is shown that for a segmentatin of $\Delta\eta\times\Delta\phi=0.022\times0.022$ both particles can be resolved in the HCAL [REF]. A series of results presented in [REFS] analyze three calorimeters benchmark configurations: HCAL(ECAL) $0.1(0.025) \eta\times5.6(1.4) deg \phi$, HCAL(ECAL) $0.05(0.012) \eta\times2.8(0.7) deg \phi$ and HCAL(ECAL) $0.025(0.006) \eta \times 1.4(0.35) deg \phi$. The jet mass resolution for jets with $p_T>3$ TeV in $t\overleftarrow{t}$ events improves by $80 \%$ and $120 \%$ for $\Delta\eta\times\Delta\phi = 0.05 \times 0.05$ and $\Delta\eta\times\Delta\phi = 0.025 \times 0.025$ cells with respect to $\Delta\eta\times\Delta\phi = 0.1\times0.1$. The resolution on the $\tau_{32}$ variable is also shown to increase as the granularity increases. In addition, eflow (particle flow) jets are shown to have a better resolution than calorimeter jets. The overlap between the distributions of the $\tau_{21}$ variable in W and QCD jets decreases from $80\%$ to $60\%$ going from $\Delta\eta\times\Delta\phi = 0.1 \times 0.1$ to $\Delta\eta\times\Delta\phi = 0.005 \times 0.005$. For $20$ TeV jets this effect is absent.



