%%%%%%%%%%%%%%%%%%%%%%%%%%%%%%%%%%%%%%%%%%%%%%%%%%%%%%%%%%%%%%%%%%%%%%%%
%                                                                      %
%     File: Thesis_Resumo.tex                                          %
%     Tex Master: Thesis.tex                                           %
%                                                                      %
%     Author: Andre C. Marta                                           %
%     Last modified :  2 Jul 2015                                      %
%                                                                      %
%%%%%%%%%%%%%%%%%%%%%%%%%%%%%%%%%%%%%%%%%%%%%%%%%%%%%%%%%%%%%%%%%%%%%%%%

\section*{Resumo}

% Add entry in the table of contents as section
\addcontentsline{toc}{section}{Resumo}

A produ\c {c}\~{a}o de pares de bos\~{o}es de Higgs \'{e} um processo chave a ser estudado em aceleradores futuros porque fornece informa\c{c}\~{a}o sobre o mecanismo de quebra da simetria eletrofraca mas est\'{a} fora to alcance do LHC.

A constru\c{c}\~{a}o de um acelerador circular com $100$ km de per\'{i}metro na \'{a}rea de Genebra est\'{a} a ser analisada pelo CERN. Espera-se que o futuro acelerador hadr\'{o}nico (FCC-hh) trabalhe a uma energia do centro de massa (CM) de $\sqrt{s}=100$ TeV e que acumule uma luminosidade integrada de $O(30)~\text{ab}^{-1}$.

Esta tese descreve um estudo de Monte Carlo que analisa o processo $hh\rightarrow b\overline{b}b\overline{b}$ numa regi\~{a}o cinem\'{a}tica de alto momento transverso, usando colis\~{o}es prot\~{a}o-prot\~{a}o a uma energia do CM de $\sqrt{s}=100$ TeV. O trabalho foca-se no impacto da granularidade do calor\'{i}metro hadr\'{o}nico na signific\^{a}ncia, $S/\sqrt{B}$. 

Para o FCC-hh, a signific\^{a}ncia atingida \'{e} $S/\sqrt{B}=8.8\pm 1.6~\text{(stat.)}~^{+4.1}_{-3.0}~\text{(sys.)}$ para uma luminosidade integrada de $\mathcal{L}=30~\text{ab}^{-1}$. Usando jatos reconstru\'{i}dos usando informa\c{c}\~{a}o do sistema de tra\c{c}os e do calor\'{i}metro, a signific\^{a}ncia varia aproximadamente $54\%$. A varia\c{c}\~{a}o \'{e} mais acentuada quando se usam jatos recontru\'{i}dos usando apenas a informa\c{c}\~{a}o do calor\'{i}metro ($\sim 71\%$).

\vfill

\textbf{\Large Palavras-chave:} Pares de bos\~{o}es de Higgs, Futuro acelerador hadr\'{o}nico circular, estudos de granularidade do calor\'{i}metro hadr\'{o}nico, FCC-hh, estado final com quatro quarks b, upgrade do detetor ATLAS

