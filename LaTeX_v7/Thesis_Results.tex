%%%%%%%%%%%%%%%%%%%%%%%%%%%%%%%%%%%%%%%%%%%%%%%%%%%%%%%%%%%%%%%%%%%%%%%%
%                                                                      %
%     File: Thesis_Results.tex                                         %
%     Tex Master: Thesis.tex                                           %
%                                                                      %
%     Author: Andre C. Marta                                           %
%     Last modified :  2 Jul 2015                                      %
%                                                                      %
%%%%%%%%%%%%%%%%%%%%%%%%%%%%%%%%%%%%%%%%%%%%%%%%%%%%%%%%%%%%%%%%%%%%%%%%

\chapter{Results}
\label{chapter:results}

In this chapter we describe the main results of the search for $hh\rightarrow b\overline{b}b\overline{b}$ at the FCC-hh using two benchmark luminosities, $3~\text{ab}^{-1}$ and $30~\text{ab}^{-1}$ (section \ref{sec:dihiggs_FCC}). The statistical analysis used to extract the signal strength and to set limits on the Higgs boson triple coupling is also discussed. In section \ref{sec:gran_studies} we show how the significance of the analysis varies as a function of the granularity of the HCAL and/or the detector configuration. We also compare the results obtained using particle flow and pure calorimeter jets.

\section{Di-Higgs discovery potential at the FCC-hh}
\label{sec:dihiggs_FCC}

%\begin{table}
%	\begin{tabular}{lcccccccccc}
%		\toprule 
%		\textbf{Selection} & SM & $\epsilon(\%)$ & 2HDM & $\epsilon(\%)$ & 4b+j & $\epsilon(\%)$& jj+0/1/2 j &$\epsilon(\%)$& $t\overline{t}$+0/1/2 j & $\epsilon(\%)$\\
%		\midrule
%		Gen level & $3.46\text{e}7$ & $100$& $5.55\text{e}7$ & $100$&$4.90\text{e}10$ & $100$&$5.47\text{e}14$ &$100$ &$2.25\text{e}12$&$100$\\
%		\rowcolor{black!10}N(b-tags)$\geq4$ & $3.20\text{e}7$& $92.5$ & $5.09\text{e}7$& $91.8$& $3.72\text{e}10$ & $75.8$ &$2.17\text{e}13$ & $3.963$&$1.20\text{e}12$& $53.5$\\
%		$p_T(j_1,j_2)\geq200$ GeV & $5.75\text{e}6$ & $16.6$ & $1.70\text{e}7$& $30.6$&$8.73\text{e}9$ & $17.8$& $4.06\text{e}12$ &$0.74$ &$2.38\text{e}10$ & $1.06$\\
%		\rowcolor{black!10}$p_T(j_1)\geq 400$ GeV & $2.99\text{e}6$ & $8.623$ & $1.01\text{e}7$& $18.2$&$3.44\text{e}9$ & $7.0$ &$1.00\text{e}12$ & $0.18$ &$1.00\text{e}10$& $0.446$\\
%		$p_T(j_2)\geq 350$ GeV & $1.98\text{e}6$ & $5.7$ & $6.22\text{e}6$& $11.2$ &$1.93\text{e}9$ &$3.9$ &$6.61\text{e}11$ &$0.121$ &$5.92\text{e}9$& $0.263$\\
%		\rowcolor{black!10}$p_T(j_1+j_2)\geq 100$ GeV & $1.61\text{e}6$& $4.648$& $4.53\text{e}6$& $8.16$&$1.62\text{e}9$& $3.3$&$3.80\text{e}11$ & $0.07$ & $5.03\text{e}9$& $0.223$\\
%		$\tau_{21}(j_1,j_2)<0.55$ & $5.91\text{e}5$ & $1.7$ &$1.85\text{e}6$ &$3.3$&$2.65\text{e}8$ & $0.54$ & $2.95\text{e}10$ & $0.005$ & $1.56\text{e}9$ & $0.069$\\
%		\rowcolor{black!10}$FW2(j_1)>0.2$ &$4.44\text{e}5$ & $1.28$& $1.50\text{e}6$& $2.7$&$1.57\text{e}8$ & $ 0.32$&$1.78\text{e}10$ & $0.003$& $4.41\text{e}8$& $0.020$\\
%		$100<M_{SD}(j1,j2)<135$ GeV & $1.46\text{e}5$&$0.422$ &$6.07\text{e}5$ & $1.09$& $6.66\text{e}6$& $0.0136$ & $4.38\text{e}8$ & $0.00008$ & $1.75\text{e}7$& $0.00078$\\
%		\bottomrule
%	\end{tabular}
%	\caption{FCC default HCAL. Entries normalized to $\mathcal{L}=30~\text{ab}^{-1}$}
%\end{table}

\begin{table}
	\centering
	\begin{tabular}{lccccc}
		\toprule 
		\textbf{Selection} & SM  & 2HDM &  4b+j & jj+0/1/2 j & $t\overline{t}$+0/1/2 j\\
		\midrule
		\multirow{2}{*}{Gen level} & $100$ & $100$ & $100$&$100$ & $100$\\
		&  ($3.46\text{e}7$) & ($5.55\text{e}7$) & ($4.90\text{e}10$) & ($5.47\text{e}14$) & ($2.25\text{e}12$)\\
		\rowcolor{black!10}N(b-tags)$\geq4$ & $92.5$ & $91.8$& $75.8$ & $3.963$& $53.5$\\
		\multirow{2}{*}{$p_T(j_1,j_2)\geq200$ GeV} & $16.6$ & $30.6$ & $17.8$ &$0.74$ & $1.06$\\ 
		& ($5.75\text{e}6$) & ($1.70\text{e}7$) & ($8.73\text{e}9$) & ($4.06\text{e}12$ ) & ($2.38\text{e}10$) \\
		\midrule \midrule
		\rowcolor{black!10}$p_T(j_1)\geq 400$ GeV & $8.623$ & $18.2$ & $7.0$ & $0.18$ & $0.446$\\ 
		$p_T(j_2)\geq 350$ GeV & $5.7$ &  $11.2$ &$3.9$ &$0.121$ & $0.263$\\
		\rowcolor{black!10}$p_T(j_1+j_2)\geq 100$ GeV &  $4.648$&  $8.16$& $3.3$& $0.07$ & $0.223$\\
		$\tau_{21}(j_1,j_2)<0.55$ & $1.7$ &$3.3$& $0.54$ & $0.005$ & $0.069$\\
		\rowcolor{black!10}$FW2(j_1)>0.2$ & $1.28$& $2.7$& $ 0.32$& $0.003$& $0.020$\\
		\multirow{2}{*}{$(100<M_{SD}(j1,j2)<135)$ GeV} & $0.422$ & $1.09$& $0.0136$ & $0.00008$ & $0.00078$\\
		&($1.46\text{e}5$)&($6.07\text{e}5$)& ($6.66\text{e}6$)& ($4.38\text{e}8$) & ($1.75\text{e}7$)\\
		\bottomrule
	\end{tabular}
	\caption{Cumulative efficiency, in percentage, of each event selection criterion for the signal samples (SM and 2HDM) and for the background sample (4b+j, jj+0/1/2 j and $t\overline{t}$+0/1/2 j). The absolute value of expected events after some key selection cuts is shown in curved brackets. The number of expected events is normalized to $\mathcal{L}=30~\text{ab}^{-1}$. The double horizontal line marks the pre-selection cuts. These results were obtained using the FCC-hh baseline detector design, as implemented in Delphes by the FCC-hh study group.}
\end{table}


\subsection{Statistical analysis}

\section{Hadronic calorimeter granularity studies for future colliders}
\label{sec:gran_studies}

In this section we present the results that allow us to compare the different detector configurations.

The softdrop mass of the leading Higgs candidate for the SM signal sample is shown on the left of figure \ref{fig:CompGran} for the different detector configurations. We see that the mass resolution increases as we increase the granularity [QUANTIFY?].

The $\tau_{21}$ variable for the leading Higgs candidate for the SM signal (filled lines) and for the $4b+j$ background (dashed lines) is shown on the right of figure \ref{fig:CompGran} for the different detector configurations. The overlap between the signal and background distributions decreases as the granularity of the HCAL increases. This means that the separation between signal and background increases. This was expected because an increase in the granularity of the hadronic calorimeter should help resolve better the substructure of boosted jets. For the $4b+j$ background the maximum overlap fraction is $0.68\pm0.12$ for the ATLAS detector and for the ATLAS HCAL. The minimum is $0.67\pm0.12$ for the remaining configurations. For the multijet background the maximum overlap is $0.55\pm0.10$ for the ATLAS detector and the minimum is $0.50\pm0.09$ for the FCC-hh default detector, with an HCAL two times less granular in $\phi$ and with an HCAL two times more granular in $\eta$ and $\phi$. For the $t\overline{t}$, the overlap is $0.78\pm0.13$ for all configurations except for the FCC-hh default configuration for which it is $0.77\pm0.13$. Regardless of the background, the overlap area between the distributions changes very little for different detector configurations. The multijet background has the smallest overlap, as expected. 

\begin{figure}
	\centering
	\begin{minipage}{.5\textwidth}
		\centering
		\includegraphics[trim={.65cm 0 0 0},clip,width=\linewidth]{./Figures/M.pdf}
		\label{fig:CompGran_M}
		%\caption{Leading Higgs candidate softdrop mass plot for the different detector configurations for the SM signal sample. This plot contains all the signal events that passed all the cuts of the baseline analysis. Note that the x axis range is from $80$ GeV to $160$ GeV in order to make the differences between the histograms more clear.}
	\end{minipage}%
	\begin{minipage}{.5\textwidth}
		\centering
		\includegraphics[trim={0 0 .65cm 0},clip,width=\linewidth]{./Figures/tau21.pdf}
		%\caption{Leading Higgs candidate $\tau_{21}$ plot for the different detector configurations for the SM signal sample. This plot contains all the signal events that passed all the cuts of the baseline analysis.}
		\label{fig:CompGran_tau21}
	\end{minipage}
	\label{fig:CompGran}
	\caption{Leading Higgs candidate softdrop mass (left) and $\tau_{21}$ (right) plots for the different detector configurations for the SM signal sample. The plots contain all the signal events that passed all the cuts of the baseline analysis. For the plot on the left note that the x axis range is from $80$ GeV to $160$ GeV in order to make the differences between the histograms more clear.}
\end{figure}

%\begin{figure}
%	\centering
%	\includegraphics[width=\linewidth]{./Figures/M.pdf}
%	\label{fig:CompGran_M}
%	\caption{Leading Higgs candidate softdrop mass plot for the different detector configurations for the SM signal sample. This plot contains all the signal events that passed all the cuts of the baseline analysis. Note that the x axis range is from $80$ GeV to $160$ GeV in order to make the differences between the histograms more clear.}
%\end{figure}
%
%\begin{figure}
%	\centering
%	\includegraphics[width=\linewidth]{./Figures/tau21.pdf}
%	\label{fig:CompGran_tau21}
%	\caption{Leading Higgs candidate $\tau_{21}$ plot for the different detector configurations for the SM signal sample. This plot contains all the signal events that passed all the cuts of the baseline analysis.}
%\end{figure}


\begin{figure}
	\centering
	\includegraphics[width=\linewidth]{./Figures/SSBplotGran.pdf}
	\label{fig:granSSB}
	\caption{Significance $(S/\sqrt{B})$ as a function of the detector configuration for $\mathcal{L}=30~\text{ab}^{-1}$ (black) and $\mathcal{L}=3~\text{ab}^{-1}$ (blue) and for particle flow jets (squares) and pure calorimeter (HCAL) jets (triangles) for the SM signal sample.}
\end{figure}

\begin{figure}
	\centering
	\includegraphics[width=\linewidth]{./Figures/EffPlotGran.pdf}
	\label{fig:granEff}
	\caption{Signal efficiency as a function of the detector configuration for particle flow jets (squares) and pure calorimeter (HCAL) jets (triangles). The error bars are drawn but are smaller than the markers.}
\end{figure}

\section{Benchmarking against ATLAS analysis at $\sqrt{s}=13$ TeV}

%\begin{figure}[!htb]
%  \centering
%  \includegraphics[width=0.25\textwidth]{Figures/Airbus_A350.jpg}
%  \caption[Caption for figure in TOC.]{Caption for figure.}
%  \label{fig:airbus1}
%\end{figure}
%
%\begin{figure}[!htb]
%  \begin{subfigmatrix}{2}
%    \subfigure[Airbus A320]{\includegraphics[width=0.49\linewidth]{Figures/Airbus_A320_sharklets.png}}
%    \subfigure[Bombardier CRJ200]{\includegraphics[width=0.49\linewidth]{Figures/Bombardier_CRJ200.png}}
%  \end{subfigmatrix}
%  \caption{Some aircrafts.}
%  \label{fig:aircrafts}
%\end{figure}
%
%Make reference to Figures \ref{fig:airbus1} and \ref{fig:aircrafts}.
%
%By default, the supported file types are {\it .png,.pdf,.jpg,.mps,.jpeg,.PNG,.PDF,.JPG,.JPEG}.
%
%See \url{http://mactex-wiki.tug.org/wiki/index.php/Graphics_inclusion} for adding support to other extensions.
%
%
%% ----------------------------------------------------------------------
%\subsubsection{Drawings}
%\label{subsection:drawings}
%
%Insert your subsection material and for instance a few drawings...
%
%The schematic illustrated in Fig.~\ref{fig:algorithm} can represent some sort of algorithm.
%
%\begin{figure}[!htb]
%  \centering
%  \scriptsize
%%  \footnotesize 
%%  \small
%  \setlength{\unitlength}{0.9cm}
%  \begin{picture}(8.5,6)
%    \linethickness{0.3mm}
%
%    \put(3,6){\vector(0,-1){1}}
%    \put(3.5,5.4){$\bf \alpha$}
%    \put(3,4.5){\oval(6,1){}}
%    %\put(0,4){\framebox(6,1){}}
%    \put(0.3,4.4){Grid Generation: \quad ${\bf x} = {\bf x}\left({\bf \alpha}\right)$}
%
%    \put(3,4){\vector(0,-1){1}}
%    \put(3.5,3.4){$\bf x$}
%    \put(3,2.5){\oval(6,1){}}
%    %\put(0,2){\framebox(6,1){}}
%    \put(0.3,2.4){Flow Solver: \quad ${\cal R}\left({\bf x},{\bf q}\left({\bf x}\right)\right) = 0$}
%
%    \put(6.0,2.5){\vector(1,0){1}}
%    \put(6.4,3){$Y_1$}
%
%    \put(3,2){\vector(0,-1){1}}
%    \put(3.5,1.4){$\bf q$}
%    \put(3,0.5){\oval(6,1){}}
%    %\put(0,0){\framebox(6,1){}}
%    \put(0.3,0.4){Structural Solver: \quad ${\cal M}\left({\bf x},{\bf q}\left({\bf x}\right)\right) = 0$}
%
%    \put(6.0,0.5){\vector(1,0){1}}
%    \put(6.4,1){$Y_2$}
%
%    %\put(7.8,2.5){\oval(1.6,5){}}
%    \put(7.0,0){\framebox(1.6,5){}}
%    \put(7.1,2.5){Optimizer}
%    \put(7.8,5){\line(0,1){1}}
%    \put(7.8,6){\line(-1,0){4.8}}
%  \end{picture}
%  \caption{Schematic of some algorithm.}
%  \label{fig:algorithm}
%\end{figure}
%
%
%% ----------------------------------------------------------------------
%\subsection{Equations}
%\label{subsection:equations}
%
%Equations can be inserted in different ways.
%
%The simplest way is in a separate line like this
%
%\begin{equation}
%  \frac{{\rm d} q_{ijk}}{{\rm d} t} + {\cal R}_{ijk}({\bf q}) = 0 \,.
%\label{eq:ode}
%\end{equation}
%
%If the equation is to be embedded in the text. One can do it like this ${\partial {\cal R}}/{\partial {\bf q}}=0$.
%
%It may also be split in different lines like this
%
%\begin{eqnarray}
%  {\rm Minimize}   && Y({\bf \alpha},{\bf q}({\bf \alpha}))            \nonumber           \\
%  {\rm w.r.t.}     && {\bf \alpha} \,,                                 \label{eq:minimize} \\
%  {\rm subject~to} && {\cal R}({\bf \alpha},{\bf q}({\bf \alpha})) = 0 \nonumber           \\
%                   &&       C ({\bf \alpha},{\bf q}({\bf \alpha})) = 0 \,. \nonumber
%\end{eqnarray}
%
%It is also possible to use subequations. Equations~\ref{eq:continuity}, \ref{eq:momentum} and \ref{eq:energy} form the Naver--Stokes equations~\ref{eq:NavierStokes}.
%
%\begin{subequations}
%    \begin{equation}
%    \frac{\partial \rho}{\partial t} + \frac{\partial}{\partial x_j}\left( \rho u_j \right) = 0 \,,
%    \label{eq:continuity}
%    \end{equation}
%    \begin{equation}
%    \frac{\partial}{\partial t}\left( \rho u_i \right) + \frac{\partial}{\partial x_j} \left( \rho u_i u_j + p \delta_{ij} - \tau_{ji} \right) = 0, \quad i=1,2,3 \,,
%    \label{eq:momentum}
%    \end{equation}
%    \begin{equation}
%        \frac{\partial}{\partial t}\left( \rho E \right) + \frac{\partial}{\partial x_j} \left( \rho E u_j + p u_j - u_i \tau_{ij} + q_j \right) = 0 \,.
%    \label{eq:energy}
%    \end{equation}
%\label{eq:NavierStokes}%
%\end{subequations}
%
%
%% ----------------------------------------------------------------------
%\subsection{Tables}
%\label{section:tables}
%
%Insert your subsection material and for instance a few tables...
%
%Make sure all tables presented are referenced in the text!
%
%Follow some guidelines when making tables:
%
%\begin{itemize}
%  \item Avoid vertical lines
%  \item Avoid “boxing up” cells, usually 3 horizontal lines are enough: above, below, and after heading
%  \item Avoid double horizontal lines
%  \item Add enough space between rows
%\end{itemize}
%
%\begin{table}[!htb]
%  \renewcommand{\arraystretch}{1.2} % more space between rows
%  \centering
%  \begin{tabular}{lccc}
%    \toprule
%    Model           & $C_L$ & $C_D$ & $C_{M y}$ \\
%    \midrule
%    Euler           & 0.083 & 0.021 & -0.110    \\
%    Navier--Stokes  & 0.078 & 0.023 & -0.101    \\
%    \bottomrule
%  \end{tabular}
%  \caption[Table caption shown in TOC.]{Table caption.}
%  \label{tab:aeroCoeff}
%\end{table}
%
%Make reference to Table \ref{tab:aeroCoeff}.
%
%Tables \ref{tab:memory} and \ref{tab:multipleColumns} are examples of tables with merging columns:
%
%\begin{table}[!htb]
%  \renewcommand{\arraystretch}{1.2} % more space between rows
%  \centering
%  \begin{tabular}[]{lrr}
%    \toprule
%                & \multicolumn{2}{c}{\underline{Virtual memory [MB]}} \\
%                & Euler       & Navier--Stokes \\
%    \midrule
%      Wing only &  1,000      &    2,000       \\
%      Aircraft  &  5,000      &   10,000       \\
%      (ratio)   & $5.0\times$ & $5.0\times$    \\
%    \bottomrule
%  \end{tabular}
%  \caption{Memory usage comparison (in MB).}
%  \label{tab:memory}
%\end{table}
%
%\begin{table}[!htb]
%  \centering
%  \renewcommand{\arraystretch}{1.2} % more space between rows
%  \begin{tabular}{@{}rrrrcrrr@{}} % remove space to the vertical edges @{}...@{}
%    \toprule
%      & \multicolumn{3}{c}{$w = 2$} & \phantom{abc} & \multicolumn{3}{c}{$w = 4$} \\
%    \cmidrule{2-4}
%    \cmidrule{6-8}
%      & $t=0$ & $t=1$ & $t=2$ && $t=0$ & $t=1$ & $t=2$ \\
%    \midrule
%      $dir=1$
%      \\
%      $c$ &  0.07 &  0.16 &  0.29 &&  0.36 &  0.71 &   3.18 \\
%      $c$ & -0.86 & 50.04 &  5.93 && -9.07 & 29.09 &  46.21 \\
%      $c$ & 14.27 &-50.96 &-14.27 && 12.22 &-63.54 &-381.09 \\
%      $dir=0$
%      \\
%      $c$ &  0.03 &  1.24 &  0.21 &&  0.35 & -0.27 &  2.14 \\
%      $c$ &-17.90 &-37.11 &  8.85 &&-30.73 & -9.59 & -3.00 \\
%      $c$ &105.55 & 23.11 &-94.73 &&100.24 & 41.27 &-25.73 \\
%    \bottomrule
%  \end{tabular}
%  \caption{Another table caption.}
%  \label{tab:multipleColumns}
%\end{table}
%
%An example with merging rows can be seen in Tab.\ref{tab:multipleRows}.
%
%\begin{table}[!htb]
%  \renewcommand{\arraystretch}{1.2} % more space between rows
%  \centering
%  \begin{tabular}{ccccc}
%    \toprule
%      \multirow{2}{*}{ABC} & \multicolumn{4}{c}{header} \\
%      \cmidrule{2-5} & 1.1 & 2.2 & 3.3 & 4.4 \\
%    \midrule
%      \multirow{2}{*}{IJK} & \multicolumn{2}{c}{\multirow{2}{*}{group}} & 0.5 & 0.6 \\
%      \cmidrule{4-5}       & \multicolumn{2}{c}{}                       & 0.7 & 1.2 \\
%    \bottomrule
%  \end{tabular}
%  \caption{Yet another table caption.}
%  \label{tab:multipleRows}
%\end{table}
%
%If the table has too many columns, it can be scaled to fit the text widht, as in Tab.\ref{tab:scale}.
%\begin{table}[!htb]
%  \renewcommand{\arraystretch}{1.2} % more space between rows
%  \centering
%  \resizebox*{\textwidth}{!}{%
%    \begin{tabular}[]{lcccccccccc}
%      \toprule
%        Variable &  a  &  b  &  c  &  d  &  e  &  f  &  g  &  h  &  i  &  j  \\
%      \midrule
%        Test 1   &  10,000 &  20,000 &  30,000 &  40,000 &  50,000 &  60,000 &  70,000 &  80,000 &  90,000 & 100,000 \\
%        Test 2   &  20,000 &  40,000 &  60,000 &  80,000 & 100,000 & 120,000 & 140,000 & 160,000 & 180,000 & 200,000 \\
%      \bottomrule
%    \end{tabular}
%  }%
%  \caption{Very wide table.}
%  \label{tab:scale}%
%\end{table}
%
%
%% ----------------------------------------------------------------------
%\subsection{Mixing}
%\label{section:mixing}
%
%If necessary, a figure and a table can be put side-by-side as in Fig.\ref{fig:side_by_side}
%
%\begin{figure}[!htb]
%  \begin{minipage}[b]{0.60\linewidth}
%    \centering
%    \includegraphics[width=\linewidth]{Figures/Bombardier_CRJ200}
%  \end{minipage}%
%  \begin{minipage}[b]{0.30\linewidth}
%    \centering
%    \begin{tabular}[b]{lll}
%      \toprule
%        \multicolumn{3}{c}{Legend} \\
%      \midrule
%        A & B & C \\
%        0 & 0 & 0 \\
%        0 & 1 & 0 \\
%        1 & 0 & 0 \\
%        1 & 1 & 1 \\
%      \bottomrule
%    \end{tabular}
%    \vspace{5em}
%  \end{minipage}
%\caption{Figure and table side-by-side.}
%\label{fig:side_by_side}
%\end{figure}

