%%%%%%%%%%%%%%%%%%%%%%%%%%%%%%%%%%%%%%%%%%%%%%%%%%%%%%%%%%%%%%%%%%%%%%%%
%                                                                      %
%     File: Thesis_Abstract.tex                                        %
%     Tex Master: Thesis.tex                                           %
%                                                                      %
%     Author: Andre C. Marta                                           %
%     Last modified :  2 Jul 2015                                      %
%                                                                      %
%%%%%%%%%%%%%%%%%%%%%%%%%%%%%%%%%%%%%%%%%%%%%%%%%%%%%%%%%%%%%%%%%%%%%%%%

\section*{Abstract}

% Add entry in the table of contents as section
\addcontentsline{toc}{section}{Abstract}

%The Electroweak Symmetry Breaking mechanism crucially depends on the value of the Higgs boson self couplings. The Higgs trilinear coupling can be probed using Higgs pair production. However, the small cross-section of this process as well as the overwhelming backgrounds make it a very challenging search. In fact, the observation of this process is most likely out of the reach of the Large Hadron Collider (LHC). Therefore, it will rely on future high energy colliders, making it a key benchmark process.
%
%In particular, the hadronic Future Circular Collider (FCC-hh), expected to work at a center of mass energy of $\sqrt{s}=100$ TeV and to deliver a total integrated luminosity of $O(30)~\text{ab}^{-1}$, is being studied by CERN to start collecting data after the High-Luminosity LHC as reached its full discovery potential, by around 2040. 

The production of pairs of Higgs bosons is a key benchmark process for future colliders because it provides crucial insight into the Electroweak Symmetry Breaking mechanism but it is out of the current reach of the Large Hadron Collider.

CERN is currently leading a study that analyzes the feasibility of a $100$ km circular collider located in the Geneva area. The hadronic Future Circular Collider (FCC-hh) is expected to work at a center of mass energy of $\sqrt{s}=100$ TeV and to collected a total integrated luminosity of $O(30)~\text{ab}^{-1}$.

This thesis describes a Monte Carlo study targeting the search for $hh\rightarrow b\overline{b}$ in a boosted kinematic regime, using proton-proton collisions at center of mass energy of $\sqrt{s}=100$ TeV. The focus is on the impact of the granularity of hadronic calorimeter on the significance, $S/\sqrt{B}$, targeting detector optimization studies for future colliders. In addition to traditional kinematic variables, jet substructure observables are also explored.

For the FCC-hh, the achieved significance is $S/\sqrt{B}=...$ for an integrated luminosity of $\mathcal{L}=30~\text{ab}^{-1}$, which is above the $5\sigma$ threshold. When using particle flow jets, the significance changes very little over the range of detector configurations considered. The change is more accentuated when using pure calorimeter jets.

\vfill

\textbf{\Large Keywords:} keyword1, keyword2,...

