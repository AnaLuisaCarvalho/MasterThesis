%%%%%%%%%%%%%%%%%%%%%%%%%%%%%%%%%%%%%%%%%%%%%%%%%%%%%%%%%%%%%%%%%%%%%%%%
%                                                                      %
%     File: Thesis_Glossary.tex                                        %
%     Tex Master: Thesis.tex                                           %
%                                                                      %
%     Author: Andre C. Marta                                           %
%     Last modified : 30 Oct 2012                                      %
%                                                                      %
%%%%%%%%%%%%%%%%%%%%%%%%%%%%%%%%%%%%%%%%%%%%%%%%%%%%%%%%%%%%%%%%%%%%%%%%
%
% The definitions can be placed anywhere in the document body
% and their order is sorted by <symbol> automatically when
% calling makeindex in the makefile
%
% The \glossary command has the following syntax:
%
% \glossary{entry}
%
% The \nomenclature command has the following syntax:
%
% \nomenclature[<prefix>]{<symbol>}{<description>}
%
% where <prefix> is used for fine tuning the sort order,
% <symbol> is the symbol to be described, and <description> is
% the actual description.

% ----------------------------------------------------------------------

\glossary{name={\textbf{ATLAS}},description={A Toroidal LHC Apparatus}}

\glossary{name={\textbf{2HDM}},description={Two Higgs Doublet Model}}

\glossary{name={\textbf{CMS}},description={Compact Muon Solenoid}}

\glossary{name={\textbf{SM}},description={Standard Model}}

\glossary{name={\textbf{BSM}},description={Beyond the Standard Model}}

\glossary{name={\textbf{BR}},description={Branching Ratio}}

\glossary{name={\textbf{CERN}},description={European Organization for Nuclear Research}}

\glossary{name={\textbf{EM}},description={Electromagnetic}}

\glossary{name={\textbf{ID}},description={Inner Detector}}

\glossary{name={\textbf{LHC}},description={Large Hadron Collider}}

\glossary{name={\textbf{LO}},description={Leading Order}}

\glossary{name={\textbf{NLO}},description={Next to Leading Order}}

\glossary{name={\textbf{HCAL}},description={Hadronic Calorimeter}}

\glossary{name={\textbf{ECAL}},description={Electromagnetic Calorimeter}}

\glossary{name={\textbf{QCD}},description={Quantum Chromodynamics}}

\glossary{name={\textbf{QED}},description={Quantum Electrodynamics}}

\glossary{name={\textbf{QFT}},description={Quantum Field Theory}}

\glossary{name={\textbf{TileCal}},description={Tile Calorimeter}}

\glossary{name={\textbf{SUSY}},description={Super Symmetry}}

\glossary{name={\textbf{VEV}},description={Vacuum Expectation Value}}