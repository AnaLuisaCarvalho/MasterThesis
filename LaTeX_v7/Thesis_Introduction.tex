%%%%%%%%%%%%%%%%%%%%%%%%%%%%%%%%%%%%%%%%%%%%%%%%%%%%%%%%%%%%%%%%%%%%%%%%
%                                                                      %
%     File: Thesis_Introduction.tex                                    %
%     Tex Master: Thesis.tex                                           %
%                                                                      %
%     Author: Andre C. Marta                                           %
%     Last modified :  2 Jul 2015                                      %
%                                                                      %
%%%%%%%%%%%%%%%%%%%%%%%%%%%%%%%%%%%%%%%%%%%%%%%%%%%%%%%%%%%%%%%%%%%%%%%%

\chapter{Introduction}
\label{chapter:introduction}

NUMBERS? (luminoisty, cross sections, BR,...)\\
-------------------------------------------------\\

THESIS GOAL

It is the ultimate goal of particle physics to discover and study all of Nature's fundamental particles and to understand their interactions. Through a joint endeavor of theorists and experimentalists, models that describe particle's dynamics and properties can be precisely probed at collider experiments such as the Large Hadron Collider (LHC). 

We know today that matter particles interact by means of four fundamental forces: electromagnetic, weak, strong and gravitational, each associated with a mediator particle. We even know that a very special particle, the Higgs boson, is responsible for generating the mass of all of these particles through a mechanism called Electroweak Symmetry Breaking (EWSB). All of this knowledge is beautifully summarized in the Standard Model of Particle Physics (SM) that was developed in the 1960's, long before many of the particles it predicts were discovered. The extraordinary precision of the predictions it delivers make it a very successful model. Its most recent prediction, the Higgs boson, was discovered in 2012 at the LHC which marks an important point in the history of particle physics: we have now found all the particles predicted by the SM and yet we know that it cannot be the whole story. Mainly because there are experimental evidences it cannot explain.

From the theoretical point of view, this is enough motivation to construct models that extend the SM but that can still deliver predictions that are compatible with experimental data. From the experimental standpoint, this is an indication that we need to keep increasing the precision of our measurements and probing new kinematic regimes in the hope of finding some discrepancy with the SM or some hint that some new phenomenon might be taking place.

A higher precision requires a larger integrated luminosity and the exploration of new kinematic regimes ask for a larger center of mass energy. Very recently, the upgrade of the LHC to its High-Luminosity (HL) version has began. It is expected to work for a period of ten years between 2026 and 2036 and it will extend the experimental reach of the LHC. In order to keep extending the physics reach of the LHC and HL-LHC, new colliders with unprecedentedly high CM energies are currently being designed in the hope that they begin to deliver data shortly after the HL-LHC has reached its full discovery potential. One of these projects is the hadronic Future Circular Collider (FCC-hh) that is expected to work at a CM energy of 100 TeV and to deliver a total integrated luminosity ten times larger than what is expected by the end of the HL-LHC operation. 

The next step for the FCC-hh project is the submission of a Conceptual Design Report by the end of 2018. This document will be used as input for the next meeting of the European Strategy for Particle Physics that will take place in the beginning of 2019. It should present a baseline design for the detector, a first cost estimate and analysis for physics benchmark processes.

Both in the HL-LHC and in future colliders, one of the most important benchmark processes is the production of pairs of Higgs bosons. Firstly, this process is predicted by the SM but has not yet been measured which is due to its very small cross section and overwhelming backgrounds. Furthermore, it provides unique insight into the EWSB mechanism because it is sensitive to the shape of the Higgs potential and can even be used to probe physics beyond the SM (BSM).

The work presented on this thesis is a Monte Carlo study that accesses the feasibility of the search for pairs of Higgs bosons at the FCC-hh in the final state with four b quarks. We choose this final state because it benefits from the large branching fraction of the Higgs boson to a pair of b quarks. However, in this channel, the SM multijet background is extremely overwhelming. Although challenging, this gives us the chance to explore the boosted kinematic regime and jet substructure observables in order to maximize the rejection of this background. We also evaluate the sensitivity of our analysis to BSM benchmark signal processes. From the detector standpoint, we evaluate how the granularity of the hadronic calorimeter influences the analysis' sensitivity and the power to resolve jet's substructure.

Chapter 2 presents an overview of the SM. It summarizes its particle content and interactions and introduces the mathematical formulation of the EWSB breaking mechanism. The successes and shortcomings of the SM are also discussed and several BSM models are introduced and their motivations discussed. Finally, a theoretical description of the production of Higgs pairs is provided.

The FCC-hh baseline accelerator and detector were highly based on LHC and its current experiments, namely ATLAS and CMS. In chapter 3, after a brief discussion of the general features of particle accelerators, we introduce the LHC and the ATLAS experiment. A discussion of jet reconstruction is included. We then introduce the FCC-hh accelerator that is expected to very similar to the LHC except larger in circumference and with more powerful magnets. The current baseline detector design for the FCC-hh is discussed and its features compared to ATLAS.  

