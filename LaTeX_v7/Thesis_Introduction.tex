%%%%%%%%%%%%%%%%%%%%%%%%%%%%%%%%%%%%%%%%%%%%%%%%%%%%%%%%%%%%%%%%%%%%%%%%
%                                                                      %
%     File: Thesis_Introduction.tex                                    %
%     Tex Master: Thesis.tex                                           %
%                                                                      %
%     Author: Andre C. Marta                                           %
%     Last modified :  2 Jul 2015                                      %
%                                                                      %
%%%%%%%%%%%%%%%%%%%%%%%%%%%%%%%%%%%%%%%%%%%%%%%%%%%%%%%%%%%%%%%%%%%%%%%%

\chapter{Introduction}
\label{chapter:introduction}

It is the ultimate goal of particle physics to discover and study all of Nature's fundamental particles and to understand their interactions. Through a joint endeavor of theorists and experimentalists, models that describe particle's dynamics and properties can be precisely probed at collider experiments such as the Large Hadron Collider (LHC). 

We know today that matter particles interact by means of four fundamental forces: electromagnetic, weak, strong and gravitational, of which the first three are associated with mediator particles. We even know that a very special particle, the Higgs boson, is linked to the mechanism that generates the mass of the mediator particles. It is called the Electroweak Symmetry Breaking (EWSB) mechanism and it involves the spontaneous breaking of a gauge symmetry. Through Yukawa couplings, the Higgs boson is also responsible for the masses of fermions. This knowledge is beautifully summarized in the Standard Model of Particle Physics (SM) that was developed in the 1960's and 1970's, long before many of the particles it predicts were discovered. The extraordinary precision of the predictions it delivers make it a very successful model. The Higgs boson, is the most recent elementary particle to be discovered in 2012 at the LHC, which marks an important point in the history of particle physics. We have now observed all the particles predicted by the SM, and yet we know that it cannot be the whole story. Mainly because there is experimental evidence of physics that it cannot explain.

From the theoretical point of view, this is enough motivation to construct models that extend the SM but that can still deliver predictions that are compatible with experimental data. From the experimental standpoint, this is an indication that we need to keep increasing the precision of our measurements and probing new kinematic regimes in the hope of finding more discrepancies with the SM or some hint that some new phenomenon is taking place.

A higher precision requires a larger integrated luminosity and a larger center of mass energy opens the door for the exploration of new kinematic regimes. Very recently, work towards the upgrade of the LHC to its High-Luminosity (HL) version has began. It is expected to work for a period of ten years between 2026 and 2036 and it will extend the experimental reach of the LHC. In order to keep extending the physics reach of the LHC and HL-LHC, new colliders with unprecedentedly high center of mass (CM) energies are currently being designed in the hope that they will begin to deliver data shortly after the HL-LHC has reached its full discovery potential. One of these projects is the hadronic Future Circular Collider (FCC-hh) that consists of a $100$ km ring located in the Geneva area and it is expected to work at a CM energy of $100$ TeV. The FCC-hh will deliver a peak luminosity of $\mathcal{L}=30\times 10^{34}~\text{cm}^{-2}\text{s}^{-1}$ in its ultimate phase. This will result in $O(30)~\text{ab}^{-1}$ per experiment which corresponds to ten times the expected integrated luminosity by the end of the HL-LHC operation. 

The next milestone for the FCC-hh project is the submission of a Conceptual Design Report by the end of 2018. This document will be used as input for the next meeting of the European Strategy for Particle Physics that will take place in the beginning of 2019. It should present a baseline design for the detector, a first cost estimate and preliminary analysis for physics benchmark processes demonstrating the physics reach of such a machine.

Both in the HL-LHC and in future colliders, one of the most important benchmark processes is the production of pairs of Higgs bosons. Firstly, this process is predicted by the SM but has not yet been measured which is due to its very small cross section and overwhelming backgrounds. Furthermore, it provides unique insight into the EWSB mechanism because it is sensitive to the shape of the Higgs potential and can also be used to probe physics beyond the SM (BSM).

In this work we study Higgs pair production at a CM energy of $\sqrt{s}=100$ TeV using the final state with four b quarks. This final state because benefits from the large branching fraction of the Higgs boson to a pair of b quarks. However, in this channel, the multijet production through Quantum Chromodynamics (QCD) interactions is overwhelming. Nonetheless, it is a well known feature of QCD interactions that the partons (and jets) produced tend to have a low transverse momentum ($p_T$). This indicates that exploring a high $p_T$ region of phase space could be the key to suppress this background. In this kinematic regime, traditional jet reconstruction techniques, that try to establish a one-to-one correspondence between partons and jets, begin to fail because the decay products of heavy and highly boosted particles become more collimated as the $p_T$ of the mother particles increase. State of the art jet reconstruction techniques make use of jets with a larger radius parameter to reconstruct both decay products (two b quarks) as a single jet that can then be used as a proxy for the original particle (a Higgs boson). In order to recover as much information as possible from these jets, it is important to analyze their intrinsic structure, referred to as substructure. 

Jet substructure techniques explore the existence of localized energy maximums inside the large-$R$ jets (subjets). For example, a jet containing the two b quarks from the decay of a Higgs boson is expected to be more consistent with having two subjets than a jet produced by a QCD process. From the standpoint of detector design for the FCC-hh, as well as for future upgrades of existing detectors, the ability to resolve the substructure of large-$R$ jets is a key requirement for which a highly granular hadronic calorimeter (HCAL) can be helpful.

The main goal of this thesis is to use boosted di-Higgs production in the four b quarks final state to study the influence of the granularity of the HCAL in significance of the detection that can be achieved.

%Although challenging, this gives us the chance to explore the boosted kinematic regime and jet substructure observables in order to maximize the rejection of this background. We also evaluate the sensitivity of our analysis to BSM benchmark signal processes. From the detector standpoint, we evaluate how the granularity of the hadronic calorimeter influences the analysis' sensitivity and the power to resolve jet's substructure.

Chapter 2 presents an overview of the SM. It summarizes its particle content and interactions and introduces the mathematical formulation of the EWSB breaking mechanism. The successes and shortcomings of the SM are presented and several BSM models are introduced and their motivations discussed. Finally, a theoretical description of the production of Higgs pairs is provided.

The FCC-hh baseline accelerator and detector were highly based on LHC and its current experiments, namely ATLAS and CMS. In chapter 3, after a brief discussion of the general features of particle accelerators, we introduce the LHC and the ATLAS experiment. A discussion of jet reconstruction is included. We then introduce the FCC-hh accelerator and its current baseline detector design. Some key detector parameters are compared to ATLAS.

In chapter 4 we present the state of the art of searches for Higgs pair production at the LHC and of feasibility studies targeting searches for this process at future colliders (High Luminosity LHC and FCC-hh). We also review previous studies that focus on the impact of the HCAL granularity on the jet mass and jet substructure observables resolution.

In chapter 5 we describe in detail the Monte Carlo samples that were used in this work. In addition, the different detector configurations that were tested are described. Chapter 6 entails the description of the analysis strategy and its optimization. A comprehensive characterization of the signal and backgrounds processes is provided. 

In chapter 7 we present and discuss the main results. We focus on the resolution of the jet mass and of the ratio of N-subjetiness variables, namely, $\tau_{21}$. In addition, we analyze how the significance changes with the granularity of the HCAL is varied. The results obtained using particle flow jets and pure calorimeter jets are compared.

Finally, conclusions are drawn in chapter 8.


