%%%%%%%%%%%%%%%%%%%%%%%%%%%%%%%%%%%%%%%%%%%%%%%%%%%%%%%%%%%%%%%%%%%%%%%%
%                                                                      %
%     File: Thesis_Conclusions.tex                                     %
%     Tex Master: Thesis.tex                                           %
%                                                                      %
%     Author: Andre C. Marta                                           %
%     Last modified :  2 Jul 2015                                      %
%                                                                      %
%%%%%%%%%%%%%%%%%%%%%%%%%%%%%%%%%%%%%%%%%%%%%%%%%%%%%%%%%%%%%%%%%%%%%%%%

\chapter{Conclusions}
\label{chapter:conclusions}

This thesis presented a feasibility study targeting the search for Higgs pair production in the $pp\rightarrow hh\rightarrow b\overline{b}b\overline{b}$ channel at a center of mass energy of $\sqrt{s}=100$ TeV. The analysis targeted a boosted kinematic regime. The impact of the granularity of the hadronic calorimeter in the achieved significance was analyzed.

The analysis consisted of a cut-based event selection designed to efficiently select signal events consistent with a boosted topology. In addition to standard kinematic variables, jet substructure variables are used to further suppress the background. The analysis required the existence of at least two b-tagged large-$R$ jets consistent with having two subjets. ... Mulijet production through QCD interactions and $t\overline{t}$ production constitute the main backgrounds.  

In addition to SM Higgs pair production, two BSM benchmark processes were considered: a $1$ TeV dark matter mediator and a heavy Higgs boson with $m_H=900$ GeV in the framework of the CP-conserving type II 2HDM, both decaying to pairs of SM Higgs bosons. The existence of heavy resonances is expected to enhance Higgs pair production with respect to the SM.

For the hadronic Future Circular Collider (FCC-hh), the significance obtained was $S/\sqrt{B}=...(...)$ for an integrated luminosity of $\mathcal{L}=30~(3)~\text{ab}^{-1}$. For $\mathcal{L}=30~\text{ab}^{-1}$, the value is above the observation threshold ($5\sigma$) which indicates that the full dataset that is expected to be collected during the operation of the FCC-hh should be enough to claim the observation of Higgs pair production, assuming SM production. For the CP-conserving type II 2HDM with $m_H=900$ GeV, the achieved significance was $S/\sqrt{B}=...(...)$ for an integrated luminosity of $\mathcal{L}=30~(3)~\text{ab}^{-1}$, which makes it an accessible benchmark for the HL-LHC and future colliders. The achieved significance for the $1$ TeV DM mediator is of order $1$ even for an integrated luminosity of $\mathcal{L}=30~\text{ab}^{-1}$, which makes it a very challenging benchmark and probably unaccessible even at future high energy colliders. 

With the ATLAS detector simulation, the achieved significance for the SM signal is $S/\sqrt{B}=...(...)$ for an integrated luminosity of $\mathcal{L}=30~(3)~\text{ab}^{-1}$. [COMPARE WITH FCC]

The resolution of jet mass and the separation provided by the $\tau_{21}$ variable are shown to increase as the granularity of the HCAL increases. This effect is more evident when using pure calorimeter jets. [NUMBERS]

When using particle flow jets, the change in the significance over the range of detector configurations tested is small. It varies from X to Y, for an integrated luminosity of $\mathcal{L}=30~\text{ab}^{-1}$. This corresponds to a X\% effect. Using pure calorimeter jets, the change in significance is more accentuated. It varies from X to Y for an integrated luminosity of $\mathcal{L}=30~\text{ab}^{-1}$, which corresponds to a X\% effect. For the same detector configuration, the significance is always smaller when using calorimeter jets. The same qualitative conclusions hold for the BSM models. The effect of the granularity is of the same order of magnitude as for the SM.

These results lead to the conclusion that the jet reconstruction performance is dominated by the tracking system. 

%BOOSTED H TO BB OBSERVATION AT THE FCC-HH \\
%- Using a simple cut based analysis we achieved a $S/\sqrt{B}$ of approximately 6 for $\mathcal{L}=30~\text{ab}^{-1}$. This is above the observation threshold (5$\sigma$). This result indicates that using the full dataset collected during the operation of the FCC-hh could lead to the observation of di-Higgs production, using the final state with four b quarks.\\
%- The sensitivity to the Higgs triple coupling is ... \\
%- For $\mathcal{L}=3~\text{ab}^{-1}$ the achieved significance is approximately 2 which is still bellow the evidence threshold (3$\sigma$). 
%\newline
%
%GRANULARITY STUDIES - SM\\
%- Based only on the plots of the Higgs invariant mass and of some substructure variables for the signal sample we can see a clear difference between the different granularity configurations. In particular, the mass resolution increases slightly and for the tau\_21 we se a shift to the left. These plots work as good safety checks and highlight the impact of the different granularity configurations. However they are based only on the signal sample and therefore do not provide any information regarding the change in significance. \\
%- Using Eflow (particle flow) jets there is not a big change in $S/\sqrt{B}$ as we change the detector configurations. The largest difference is of ~X \%. \\
%- Based on this observation we redid the analysis using pure calorimeter jets (reconstructed using only information from the HCAL). The difference in $S/\sqrt{B}$ over the detector configurations tested increases. The largest difference is of ~Y \%. \\
%- Eflow jets use information from both the tracking and the hadronic calorimeter (in the case on hadronic jets). Based on the fact that $S/\sqrt{B}$ varies very little when using eflow jets and that its variation increases if we use calorimeter jets we conclude that the resolution of the tracking system is so high that it dominates the jet reconstruction.\\
%- Nonetheless the signal efficiency (computed as the ratio between the number of signal events after all the analysis cuts and the total number of events) increases as the granularity of the detector increases when using eflow and calorimeter jets. This increment is more accentuated in the case of calorimeter jets where the maximum difference is ~Z \%. For Eflow jets the maximum variation is ~W \%. \\
%\newline
%
%GRANULARITY STUDIES - BSM\\
%- In addition to the SM production of Higgs pairs we also analyzed two benchmark models in which new heavy particles couple to the SM Higgs boson through the s-channel diagram.
